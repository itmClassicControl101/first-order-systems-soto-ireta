\documentclass{report}
\usepackage[utf8]{inputenc}
\usepackage{graphicx}
\usepackage{amsmath, amssymb}
\usepackage{pgfplots}
\usepackage{listings}
\usepackage{enumerate}
\usepackage{tikz}
\usepackage{float}
\usepackage[spanish]{babel}
%\usepackage[backref=page]{hyperref}
\usepackage[hidelinks]{hyperref} 
\lstset{language=Scilab, breaklines=true, basicstyle=\footnotesize}
\lstset{numbers=left, numberstyle=\tiny, stepnumber=2, numbersep=-8pt}

\setcounter{secnumdepth}{0} 
\setcounter{tocdepth}{1} 
\newcounter{ns}
\addtocounter{ns}{1} 
\graphicspath{{figuras/}}

\renewcommand{\figurename}{Figura}

\begin{document}
	
	\begin{titlepage}
		
		\begin{center}
			\vspace*{-1in}
			\begin{figure}[htb]
				\begin{center}
					\includegraphics[width=8cm]{logo}
				\end{center}
			\end{figure}
			
		\begin{large}
				Instituto Tecnologico de Morelia\\
				"Jose Maria Morelos y Pavon"\\
		\end{large}
			\vspace*{0.15in}
		\begin{large}
			Departamento de Ingeniería Electrónica\\
		\end{large}
			\vspace*{0.4in}
			
			\begin{Large}
				\textbf{Reporte de Practica No. 1} \\
			\end{Large}
			\vspace*{0.3in}
			\begin{large}
				M.C: Gerardo Marx Chavez-Campos.\\
				Materia: Control 1.\\
				Alumnos:\\
				 		David Ireta Cazarez No. Ctrl. 13120127.\\
						Edgar Soto Martinez No. Ctrl. 12121155.\\
				Fecha: 27/10/2017\\
			\end{large}
			\rule{80mm}{0.1mm}\\
			\vspace*{0.1in}
		\end{center}
		
	\end{titlepage}

%___________Termina Portada______________________________________________
	

	
	\chapter*{}
	\section{Introducción}
	La función de tansferencia de un sistema de primer orden se caracteriza por tener el polinomio del denominador de primer grado. En funcion de como sea el numerador:\\
	
	1.- SISTEMA DE PRIMER ORDEN SIN CERO:\\
	Tiene una constante como numerador.
	
	\begin{equation*}
		G(s) = \frac{Y(s)}{U(s)}=\frac{K}{1+\tau s}
	\end{equation*}
	
	Los parámetros que aparecen en la función de transferencia son la ganancia K, que coincide con la ganancia estática G(0), y la constante de tiempo $\tau$. La ganancia estática K puede tener cualquier signo, sin embargo para que este sistema sea estable se debe cumplir que $\tau>0$. \\
	
	2.- SISTEMA DE PRIMER ORDEN CON CERO NULO.
	
	\begin{equation*}
		G(s) = \frac{Y(s)}{U(s)}=\frac{Ks}{1+\tau s}
	\end{equation*}
	
	Los parámetros que aparecen en la funcion de transferencia son la ganancia K, que en este caso NO coincide con la ganancia estática G(0) que es igual a 0, y la constante de tiempo $\tau$. La ganancia K puede tener cualquier signo, mientras que $\tau$ debe ser positivo para que el sistema sea estable.\\
	
	3.- SISTEMA DE PRIMER ORDEN CON CERO NO NULO.
	
	\begin{equation*}
		G(s) = \frac{Y(s)}{U(s)}=\frac{k(1+Ts)}{1+\tau s}
	\end{equation*}
	
	En este último caso, los parámetros son la ganancia K, que vuelve a coincidir con la ganancia estática G(0) como en el sistema de primer sin cero, la constante de tiempo T asociada al cero y la constante de tiempo $\\tau$ asociada al polo o constante de tiempo del sistema. La ganancia K y la constante de tiempo T pueden tener cualquier signo, mientras que $\tau$ debe ser positivo para que el sistema sea estable.
	
	
			
		\chapter*{METODOLOGÍA}
	\section{ANÁLISIS DE UN SISTEMA DE PRIMER ORDEN}
	
	DESARROLLO TEÓRICO\\
	
	Tenemos el siguiente sistema hidráulico de entrada $W_i (t)$ y la salida $W_o (t)$ en la figura 1.
	
	

	La ecuación diferencial que define este sistema es la siguiente de la cual depende de la altura (h(t)). ???
	
	\begin{equation*}
		\omega_i(t)=A\frac{dh(t)}{d(t)}+R\frac{h(t)}{rg}
	\end{equation*}
	
	Después proponemos el circuito eléctrico de corriente del sistema hidráulico y obtuvimos lo siquiente en la figura 2:
	
	
	 
	Entonces en la ecuación de nuestra analogia electrica es:
	
	\begin{equation*}
		 \omega_i(t)=C\frac{dh(t)}{dt}+\frac{h(t)}{R}
	\end{equation*}
	
	\begin{equation*}
		Donde:	R=\frac{R}{rg}\quad y \quad C=A
	\end{equation*}
	
	Usamos la Transformada de Laplace.
	
	\begin{equation*}
		\omega_i(t)=CsH(s) + \frac{H(s)}{R}
	\end{equation*}
	
	Factorizamos la ecuacion anterior.
	
	\begin{equation*}
		\omega_i(s)=H(s) \left[Cs+\frac{1}{R}\right]
	\end{equation*}
	
	Despejando H(s) y factorizando H(s) tenemos la función de transferencia.
	
	\begin{equation*}
		\frac{H(s)}{\omega_i(s)}=\frac{\frac{1}{C}}{s+\frac{1}{RC}}
	\end{equation*}
	
	La función de transferencia se puede expresar en su forma general.
	
	\begin{equation*}
		G(s)=\frac{bs+c}{ds+a}
	\end{equation*}
	
	\begin{equation*}
	Donde:
		a=\frac{1}{RC},\quad b=0,\quad c=\frac{1}{C},\quad d=1
	\end{equation*}
	
	Sabemos que la respuesta del sistema es de la siguiente forma.
	
	\begin{equation*}
		Y(s)=\frac{1}{s}\left(\frac{bs+c}{ds+a}\right)
	\end{equation*}
	
	Una vez obtenido esto aplicamos fracciones parciales y la transformada inversa de Laplace obtenemos lo siguiente.
	
	\begin{equation*}
		y(t)=\frac{c}{a}u_s(t)+\left(\frac{b}{a}-\frac{c}{a}\right)e^\frac{-at}{d}u_s(t)
	\end{equation*}
	
	CODIGO\\
	
	Acontinuación mostraremos el código para graficar la respuesta del sistema en el dominio del tiempo utlizando la ecuación anterior.
	
	\begin{lstlisting}[frame=single]
	function[y] = funcion(a,b,c,d)//se establece la salida y los parametros de a,b,c,d
	t = 0:0.1:20 //vector de tiempo de 0 a 20 con incrementos de 0.1
	y = (a/c)+(((b*c)-(a*d))/(d*c))*exp((-a*t)/d);//Respuesta del sistema en el tiempo
	plot(t,y)//Funcion de graficacion
	endfunction
	\end{lstlisting}
		
	Para obtener los 3 casos mencionados en la práctica se realizaron los siguientes cálculos.\\
	
	CASO 1:\\
	
	$\omega_i(t) > \omega_o(t)$, tomamos la siguiente ecuación:
	
	\begin{equation*}
		\omega_i(s)=H(s)\left[Cs+\frac{1}{R}\right]
	\end{equation*}
		
	Dónde tomamos $\omega_i(t)$ como entrada y $\frac{h(t)}{R}$ como salida, proponemos un valor de 2 para la entrada y 1 para la salida, obteniendo como resultado lo siguiente:
	
	\begin{equation*}
		\omega_i(t)=CsH(s)+\frac{H(s)}{R}
	\end{equation*}
	
	\begin{equation*}
		2=x+1
	\end{equation*}
				
	\begin{equation*}
		x=1
	\end{equation*}		
	
	Entonces en la ecuación siguiente:
	
	\begin{equation*}
		G(s)=\frac{bs+c}{ds+a}
	\end{equation*}
	
	\begin{equation*}
		Donde:
		a=\frac{1}{RC},\quad b=0,\quad c=\frac{1}{C},\quad d=1
	\end{equation*}
		
	a es la salida y es igual a 1 (a=1), el valor del capacitor lo tomaria d=1, como no hay ninguna s multiplicando a b es igual a 0 (b=0) y c que es el otro numerador es igual a 1 (c=1).\\
	
	CASO 2\\
	
	$\omega_i(t) < \omega_o(t)$ proponemos el valor de 1 para la entrada 2 para la salida, quedando la ecuacion de la siguiente forma:
	
	\begin{equation*}
		1=x+2
	\end{equation*}
	
	\begin{equation*}
		x=-1
	\end{equation*}	
	
	Entonces en la ecuación siguiente:
	
	\begin{equation*}
	G(s)=\frac{bs+c}{ds+a}
	\end{equation*}
	
	\begin{equation*}
	Donde:
	a=\frac{1}{RC},\quad b=0,\quad c=\frac{1}{C},\quad d=1
	\end{equation*}
	
	a será el valor de salida (a=2), el valor del capacitor lo toma d=1 y como no hay s mulitplicando a b por lo tanto b=0 y c=1.\\
	
	CASO 3\\
	
	$\omega_i(t) = \omega_o(t)$ se da el valor 1 a todos los coeficientes ya que de esta manera se evita la indeterminacion en la siguiente ecuacion:
	
	\begin{equation*}
	y(t)=\frac{c}{a}u_s(t)+\left(\frac{b}{a}-\frac{c}{a}\right)e^\frac{-at}{d}u_s(t)
	\end{equation*}
	
	\chapter*{RESULTADOS Y OBSERACIONES}
	
	Ahora mostraremos la respuesta al impulso del sistema para los casos indicados en la practica usando la funcion de transferencia desarrollada por nosotros como la funcion proporcionada por el profesor.\\
	
	\begin{figure}[H]
		\centering
		\includegraphics[width=6cm]{Figura1}
		\caption{Respues del sistema en $\omega_i > \omega_o$ en base a la funcion de transferencia del profesor}
		\label{fig:figura7}
	\end{figure}		

	\begin{figure}[H]
		\centering
		\includegraphics[width=6cm]{Figura2}
		\caption{Respues del sistema en $\omega_i < \omega_o$ en base a la funcion de transferencia del profesor}
		\label{fig:figura7}
	\end{figure}
	
	\begin{figure}[H]
		\centering
		\includegraphics[width=6cm]{Figura3}
		\caption{Respues del sistema en $\omega_i = \omega_o$ en base a la funcion de transferencia del profesor}
		\label{fig:figura7}
	\end{figure}
	
	Mostramos el codigo generado en Scilab para la función de transferencia proporcionada por el profesor.\\
	
	\begin{lstlisting}[frame=single]
	s= %s
	poly (0 , 's')
	k = 1;
	Tau = 1;
	simpleSys=syslin('c', k/(1+Tau*s))
	x=0:0.001:10;
	y=csim('step', x, simpleSys)
	plot(x,y)
	\end{lstlisting}
	
	\begin{figure}[H]
		\centering
		\includegraphics[width=6cm]{Figura4}
		\caption{Respues del sistema en $\omega_i > \omega_o$ en base a la funcion de transferencia del profesor}
		\label{fig:figura7}
	\end{figure}

	\begin{figure}[H]
		\centering
		\includegraphics[width=6cm]{Figura6}
		\caption{Respues del sistema en $\omega_i = \omega_o$ en base a la funcion de transferencia del profesor}
		\label{fig:figura7}
	\end{figure}

	\begin{figure}[H]
		\centering
		\includegraphics[width=6cm]{Figura3}
		\caption{Respues del sistema en $\omega_i = \omega_o$ en base a la funcion de transferencia del profesor}
		\label{fig:figura7}
	\end{figure}
	
	\chapter*{CONCLUSIONES Y OBSERVACIONES}

	
OBSERVACIONES\\

	DAVID IRETA CAZAREZ\\
Cuando simulamos el codigo en el programa de Scilab, observamos los comportamientos del sistema hidráulico de nuestra practica. Cuando la entrada es mayor a la salida pudimos observar el comportamiento del capacitor cuando esta en carga, asi mismo, cuando la salida es mayor a la entrada vimos la descarga analogica del capacitor y al ser iguales no la pudimos observar porque la respuesta es constante al ser iguales y nos da como resultado una linea recta.\\

	EDGAR SOTO MARTINEZ\\
Se verificó el funcionamiento del sistema porporcionado en la práctica y poder visualizar la respuesta del mismo dependiendo de los diferentes casos que se nos pedian y de esa manera poder entender el comportamiento y entender de una mejor manera el tema\\

CONCLUSIONES\\

	DAVID IRETA CAZAREZ.\\
lo que pudimos observar fue que para obtener la respuesta d eun sistema de primer orden como la que hicimos en esta practica en cualquier sistema podra tener respues en el dominio del tiempo ya que es independiente de sus coeficientes.\\ 

	EDGAR SOTO MARTINEZ\\
Pudimos observar el comportamiento del sistema, fue complicada la práctica debido a que se esta comenzando a utilizar latex por lo cual se complico un poco mas el poder llevar a cabo el reporte.
			
\end{document}
